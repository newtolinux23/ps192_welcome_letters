\documentclass[12pt,letterpaper]{article}
\usepackage{pdfpages}
\usepackage{fancyhdr}
\usepackage[colorlinks=true, urlcolor=blue, linkcolor=blue]{hyperref}
\usepackage{graphicx}
\usepackage[top=1.4in, left=0.5in, right=0.5in, bottom=0.8in]{geometry}
\usepackage[T1]{fontenc}
\usepackage{helvet}
\usepackage{booktabs}
%\usepackage[table,xcdraw]{xcolor}
\pagestyle{fancy}
\renewcommand{\headrulewidth}{0pt}
\renewcommand{\footrulewidth}{0pt}
\setlength{\parindent}{0em}
\setlength{\parskip}{1em}


\fancyfoot[C]{\setlength{\unitlength}{1in}\begin{picture}(5,0)\put(-1.8,-1){\includegraphics[width=8.8in,height=1.3in]{logo-1}}\end{picture}}
\fancyhead[C]{\setlength{\unitlength}{1in}\begin{picture}(5,0)\put(-1.9,-1){\includegraphics[width=8.9in,height=1.3in]{logo-2}}\end{picture}}

\pagenumbering{gobble}
\addtolength{\evensidemargin}{-2in}
\addtolength{\topmargin}{-0.5in}
\addtolength{\textwidth}{0in}
%%%%%%%%%%%%%%%%%%%%%%%%%%%%%%%%%%%%%%%%%%%%%%%%%%%%%%%%%%%%%%%%%%

\begin{document}
\vspace*{0.5in}
Date: \href{https://www.ps192.org/apps/bbmessages/show_bbm.jsp?REC_ID=139439}{14 de septiembre de 2023} 

\textbf{Asunto: Criterios de promoción y política de calificaciones 2023-24}

Queridos padres y guardianes,

La Disposición del Canciller A-501 implementa una política de promoción en todo el sistema con estándares de promoción claramente definidos para cada grado. El P.D. 192 Política de Criterios de Promoción proporciona el proceso y los procedimientos para la implementación de esta política de promoción. Esta política entra en vigor a partir del 7 de septiembre de 2023. 

Esta política se promulga en el contexto de los siguientes objetivos establecidos por la Disposición del Canciller A-501:

Todos los estudiantes desde jardín de infantes hasta quinto grado cumplirán o excederán estándares académicos rigurosos en un plan de estudios básico basado en el desempeño. En los grados 3 a 5, todos los estudiantes cumplirán o excederán los estándares de promoción mencionados en este regulación, y establecido en la
guía emitida por el DOE, con el fin de ser promovido al el próximo grado y, en
última instancia, estar preparados para la universidad y las carreras.
\begin{itemize}
\item Toda la comunidad escolar participará continuamente en la creación y el apoyo de estrategias efectivas para mejorar el rendimiento estudiantil.
\item Se utilizará de forma continua un sistema integral de evaluación de estudiantes, alineado con los estándares de desempeño establecidos por el estado y la ciudad, para medir el progreso de los estudiantes y mejorar la instrucción en el aula.
\end{itemize}
% Please add the following required packages to your document preamble:
% \usepackage{booktabs}
\begin{table}[h]
\begin{center}
\begin{tabular}{@{}|l|l|@{}}
\toprule
\multicolumn{2}{|c|}{\textbf{Sistema de calificación del trabajo en clase}} \\ \midrule
Evaluaciones internas              & 50 por ciento          \\ \midrule
Trabajo de clase diario            & 30 por ciento          \\ \midrule
Participación en el aula           & 10 por ciento          \\ \midrule
Proyectos                          & 5 por ciento           \\ \midrule
Tarea                              & 5 por ciento           \\ \bottomrule
\end{tabular}
\end{center}
\end{table}
Criterios de promoción para los grados K-2:
\begin{itemize}
\item 95 por ciento de asistencia
\pagebreak
\pagebreak
\vspace*{1.5cm}
\item Cumplir con los estándares de desempeño en TODAS las materias básicas: ELA, Matemáticas, SS y Ciencias. Esto significa obtener un Nivel de Desempeño 2 (una puntuación numérica del 65 por ciento) en todas las áreas temáticas básicas: Lectura, Escritura, Matemáticas, Ciencias y Estudios Sociales. Se utilizará el promedio de las pruebas y exámenes de las unidades.
para determinar la calificación general:
	\begin{itemize}
	\item Level 1: Una puntuación media agregada de 0 a 64 puntos.
	\item Level 2: Una puntuación media agregada de 65 a 79 puntos.
	\item Level 3: Una puntuación media agregada de 80 a 89 puntos.
	\item Level 4: Una puntuación media agregada de 90 a 100 puntos.
	\end{itemize}
\end{itemize}	

Lectura: Cumplir con el punto de referencia de lectura DRA mínimo específico para el grado\begin{itemize}
\item Punto de referencia de lectura de jardín de infantes nivel 6 (E)
\item Primer grado: Nivel de referencia de lectura 15-16 (L)
\item Segundo grado: Nivel de referencia de lectura 18 (J)
\end{itemize}

Escritura: Obtenga una calificación de desempeño acumulativa de Nivel 2 en el Portafolio de Escritura
\begin{itemize}
\item Jardín de infantes: 4 piezas de escritura (2 de ficción y 2 de no ficción)
\item Primer grado: 4 tareas de interpretación de escritura (2 de ficción y 2 de no ficción)
\item Segundo grado: 4 tareas de interpretación de escritura (2 de ficción y 2 de no ficción)
\end{itemize}

Matemáticas: Obtenga una calificación de desempeño acumulativa de Nivel 2. El promedio de las pruebas y exámenes de las unidades se utilizará para determinar la calificación general.
\begin{itemize}
\item Nivel 1: una puntuación media agregada de 0 a 64 puntos
\item Nivel 2: Una puntuación media agregada de 65 a 79 puntos
\item Nivel 3: Una puntuación media agregada de 80 a 89 puntos
\item Nivel 4: Una puntuación media agregada de 90 a 100 puntos
\end{itemize}
\pagebreak
\pagebreak
\vspace*{1.5cm}
Asignaciones de proyectos: Obtenga una calificación de desempeño acumulativa de Nivel 2 en cada proyecto.
\begin{itemize}
\item Jardín de infantes: 3 proyectos individuales (diciembre – S.S.; febrero – Matemáticas; abril – Ciencias)
\item Primer grado: 3 proyectos individuales (diciembre – S.S.; febrero – Matemáticas; abril – Ciencias)
\item Segundo grado: 3 proyectos individuales (diciembre – S.S.; febrero – Matemáticas; abril – Ciencias)
\end{itemize}

Recomendación del maestro
\begin{itemize}
\item Análisis holístico y evidencia del trabajo en clase.
\end{itemize}

Criterios promocionales para los grados 3-5
\begin{itemize}
\item 95 por ciento de asistencia
\item Cumplir con los estándares de desempeño en TODAS las materias básicas: ELA, Matemáticas, SS y Ciencias. Esto significa obtener un Nivel de Desempeño 2 (una puntuación numérica del 65 por ciento) en todas las áreas temáticas básicas: Lectura, Escritura, Matemáticas, Ciencias y Estudios Sociales. Se utilizará el promedio de las pruebas y exámenes de las unidades para determinar la calificación general:
	\begin{itemize}
	\item Nivel 1: un promedio agregado de 0 a 64 puntos
	\item Nivel 2: un promedio agregado de 65 a 79 puntos
	\item Nivel 3: un promedio agregado de 80 a 89 puntos
	\item Nivel 4: un promedio agregado de 90 a 100 puntos
	\end{itemize}
\end{itemize}

Lectura: Cumplir con el punto de referencia de lectura DRA mínimo específico para el grado
\begin{itemize}
\item Tercero: Nivel de lectura de referencia 34-38 (M-N)
\item Cuarto: Grado: Nivel de lectura de referencia 38-40 (O-P)
\item Quinto: Grado: Nivel de referencia de lectura 50 (Q-R)
\end{itemize}
\pagebreak
\vspace*{1.5cm}
Escritura: Obtenga una calificación de desempeño acumulativa de Nivel 2 en el Portafolio de Escritura
\begin{itemize}
\item Tercero: 4 piezas de escritura (2 de ficción y 2 de no ficción)
\item Cuarto grado: 4 tareas de interpretación de escritura (1 de ficción y 3 de no ficción)
\item Quinto grado: 4 tareas de interpretación de escritura (1 de ficción y 3 de no ficción)
\end{itemize}

Matemáticas: Obtenga una calificación de desempeño acumulativa de Nivel 2. El promedio de las pruebas y exámenes de las unidades se utilizará para determinar la calificación general.
\begin{itemize}
\item Level 1: Una puntuación media agregada de 0 a 64 puntos.
\item Level 2: Una puntuación media agregada de 65 a 79 puntos.
\item Level 3: Un promedio agregado de puntuación de 80 a 89 puntos.
\item Level 4: Un promedio agregado de puntuación de 90 a 100 puntos.
\end{itemize}

Asignaciones de proyectos: Obtenga una calificación de desempeño acumulativa de Nivel 2 en cada proyecto.
\begin{itemize}
\item Tercer grado: 3 proyectos individuales (diciembre – S.S.; febrero – Matemáticas; abril – Ciencias)
\item Cuarto grado: 3 proyectos individuales (diciembre – S.S.; febrero – Matemáticas; abril – Ciencias)
\item Quinto grado: 3 proyectos individuales (diciembre – S.S.; febrero – Matemáticas; abril – Ciencias)
\end{itemize}
Recomendación del maestro
\begin{itemize}
\item Análisis holístico y evidencia del trabajo en clase.
\end{itemize}

Criterios promocionales para estudiantes del idioma inglés

Los estudiantes que aprenden inglés deberán cumplir con los estándares promocionales según la cantidad de años en las Escuelas Públicas de la Ciudad de Nueva York:
\pagebreak
\vspace*{1.5cm}
\begin{itemize}
\item ELL y SIFE de primer año
	\begin{itemize}
	\item Cumplir con los puntos de referencia en áreas temáticas específicas como Matemáticas, SS y Ciencias en su idioma nativo.
	\end{itemize}
\item ELL de 2do y 3er año
	\begin{itemize}
	\item Obtenga un nivel 2 en la Evaluación de Matemáticas del Estado de Nueva York y obtenga los avances esperados en el NYSESLAT (51 puntos dentro de un nivel de competencia)
	\item Obtener al menos una puntuación del 65 por ciento (Nivel de rendimiento 2) en un mínimo de tres áreas temáticas básicas.
	\end{itemize}
\item Los estudiantes ELL de cuarto año cumplirán con los mismos estándares que los estudiantes con dominio del idioma inglés.
\end{itemize}

Criterios de promoción para estudiantes de educación especial\begin{itemize}
\item Los estudiantes de Educación Especial deberán cumplir con los estándares de promoción establecidos en el IEP del estudiante.
\item Un estudiante cuyo IEP no especifica criterios de promoción modificados se sujetará a los mismos criterios de promoción estándar que los estudiantes de educación general.
\item Los maestros utilizarán todas las evaluaciones disponibles: pruebas estandarizadas,
tareas de desempeño, evaluaciones continuas del trabajo de los estudiantes, notas de conferencias, observaciones de los maestros y juicio profesional, como un mecanismo para mejorar la instrucción en el aula y brindar a los padres información detallada sobre el progreso académico de sus hijos.
\end{itemize}
Todos los criterios promocionales están sujetos a la aprobación final del Director. Los padres también participarán en el proceso de toma de decisiones. Los maestros mantendrán colecciones del trabajo de los estudiantes y datos formativos y sumativos que documenten el progreso de los estudiantes hacia el cumplimiento de los estándares y puntos de referencia de desempeño. Los maestros se reunirán con los padres regularmente para:
\begin{itemize}
\pagebreak
\vspace*{1.5cm}
\item Nuestro personal empleará varios métodos de comunicación para garantizar que los padres y tutores estén constantemente informados sobre el desarrollo socio-emocional y académico de sus hijos.
	\begin{itemize}
	\item Conferencias virtuales de Zoom o Google
	\item Conversaciones telefónicas
	\item Se utilizará comunicación escrita, que incluye ClassDojo, correo electrónico y mensajes de texto, para informar a los padres.
	\end{itemize}
\end{itemize}


En unidad,

\includegraphics[width=0.2\textwidth]{hil_signature}

\textbf{Hilduara Abreu}

\textbf{Principal}

\textit{¡La escuela donde el aprendizaje es divertido!}

\url{www.ps192.org}

\end{document}
