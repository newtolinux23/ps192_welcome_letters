\documentclass[14pt, letterpaper, twoside]{article}
\usepackage[hmarginratio=1:1, margin=1in]{geometry}
\usepackage{fancyhdr}
\usepackage{titlepic}
\usepackage{pdfpages}
\usepackage[colorlinks=true, urlcolor=blue, linkcolor=blue]{hyperref}
\usepackage{graphicx}
\usepackage[mmddyyyy]{datetime}
\usepackage{fancyhdr}
\setlength{\parskip}{2mm}
\setlength{\parindent}{0mm}
\setcounter{secnumdepth}{8}
\setcounter{tocdepth}{8}
\usepackage{setspace}
\pagestyle{fancy}
\fancyfoot{}
\usepackage{times}
\usepackage{hyperref}


\titlepic{\includegraphics[width=2in]{logo.png}}
\title{\textbf{Creating a Classroom Mission Statement}
    \begin{spacing}{1.5} 
	  This activity accomplishes the goal to impar valuable life skills, character development, and
	  a strong sense of belonging and unity within school community. It encourages students to become active participants in shaping our school's identity and culture, emphasizing the importance of a class mission statement, teamwork, and leadership.
    \end{spacing}}
  \date{}
\author{September 11, 2023}


% Cover Page
%%%%%%%%%%%%%%%%%%%%%%%%%%%%%%%%%%%%%%%%%%%%%%%%%%%%%%%%%%%%%%%%%%%%%%%%%%%%%%%%%%%%%%%%%%%%%%%%%%
\begin{document}
\includepdf[pages=1,fitpaper]{ps192-mascot}

\begin{titlepage}
  \maketitle 
  \thispagestyle{empty} 
\end{titlepage}

\pagenumbering{arabic}
\pagestyle{headings}

\fancyhf{}
\fancyhead[L]{\textit{Creating a Classroom Mission Statement}}
\fancyhead[R]{\thepage}
\fancyfoot[C]{The School of Joyful Learning!}
\pagestyle{fancy}
\renewcommand{\footrulewidth}{1 px}

% TOC
%%%%%%%%%%%%%%%%%%%%%%%%%%%%%%%%%%%%%%%%%%%%%%%%%%%%%%%%%%%%%%%%%%%%%%%%%%%%%%%%%%%%%%%%%%%%%%%%%%

\newpage
\tableofcontents

\thispagestyle{empty}
\newpage


\section{Lesson Plan: Creating a Classroom Mission Statement Grades 3-5}
Lesson Objective:

In this engaging and interactive activity, students embark on a journey to define the essence and purpose of their classroom by creating a meaningful mission statement. The activity is designed to foster teamwork, self-reflection, and an understanding of our core values, all while aligning with the PS 192 mission statement. Read aloud and analyze the lesson objective to co-create the "I can statement..."
\begin{itemize}
	\item Setting the stage: Brainstorming (15 minutes)
				\begin{itemize}
					\item Students are asked to work independently. Review and set the expectation for the independent work routine. Provide feedback in real time before sending the student to work by
								themselves. The teacher reads aloud and co-analyze the \href{https://www.ps192.org/apps/pages/index.jsp?uREC_ID=1504973&type=d&pREC_ID=1646782}{P.S. 192 Mission Statement}
					\item Through this process, students engage in self-awareness as they contemplate the qualities that define their classroom community: What makes their classroom unique and special?. Each student
								will write 1-2 characterists in grades K-1, 3-4 characteristics in grades 2-3 and 4-5 characteristics in grades 4-5. Students will use their writing journal or notebook to record their
								answers. Reference the notebook criteria.
				\end{itemize}
	\item Collaborative Crafting (10 minutes)
				\begin{itemize}
					\item Students begin by brainstorming ideas for their classroom's mission statement.They reflect on what makes their school and classroom unique and special, considering the
								\href{https://www.ps192.org/apps/pages/index.jsp?uREC_ID=1504973&type=d&pREC_ID=1646782}{school mission statement}. Divide students into small groups to collaborative draft their class
								mission statement using the characteristics each individual member of the group provided. Team must collaborate and come to a consensus in selecting the characterists to co-create their
								class mission statement. Each group designates a spokesperson to present their ideas to the whole class.
				\end{itemize}
	\item The Voting and Refinement Process (10 minutes)
				\begin{itemize}
					\item The voting process teaches students about democratic decision-making. They will experience the importance of individual voices and the collective will of the community in co-drafting
								the class mission statement.
					\item As a class, in collaboration with the teacher, students discuss and refine the ideas shared by different groups. They identify common themes and values that resonate with everyone.
								Reference and practice the timer and transition routine.
					\item During this collaborative crafting of the mission statement, students practice relationship skills, as they learn to navigate differing opinions and compromise on shared values.
					\item Respectful and responsible decision-making is encouraged as the class collectively shapes their mission statement.
				\end{itemize}
	\item Mission Statement Presentation (5 minutes)
				\begin{itemize}
					\item The finalized classroom mission statement is presented to the entire class.
					\item Students are guided to connect their mission statement with the PS192 mission statement, reinforcing the idea of a larger community to which they belong.
					\item This final step reinforces a sense of identity and purpose within the classroom.
				\end{itemize}
	\item Conclusion (5 minutes)
				\begin{itemize}
					\item In the concluding discussion, students reflect on the social-emotional skills they employed throughout the activity, such as teamwork, empathy, and respect. They understand that
								these skills are not only essential for crafting a meaningful mission statement but also for building a harmonious and supportive classroom environment. The importance of upholding the
								mission statement in their daily interactions is emphasized. This activity not only cultivates a sense of belonging and purpose among elementary school students but also equips them with
								crucial social-emotional skills that will serve them well in their academic journey and beyond. It reinforces the idea that their classroom is not just a physical space but a community
								where they can learn, grow, and thrive together.
				\end{itemize}
	\item Assessment:
				\begin{itemize}
					\item Assess students based on their participation in group discussions, their ability to collaborate, and their understanding of the concept of a mission statement.
					\item Consider using a rubric that includes criteria like teamwork, creativity, and alignment with the school's mission statement.
				\end{itemize}
	\item Extension Activity
				\begin{itemize}
					\item Have students create a visual representation (poster or artwork) of the classroom mission statement to display in the classroom.
					\item Note: Adapt the lesson plan as needed to suit the specific grade level and needs of your students.
				\end{itemize}
	\item Homework
				\begin{itemize}
					\item Encourage students to write or draw in their individual journals about how they will contribute to living up to the classroom mission statement.
				\end{itemize}
\end{itemize}
\newpage

\section{Lesson Plan: Creating a Classroom Mission Statement Grades K-2}
Lesson Objective:

In this engaging and interactive activity, students embark on a journey to define the essence and purpose of
their classroom by creating a meaningful mission statement. The activity is designed to foster teamwork,
self-reflection, and an understanding of our core values, all while aligning with the PS 192 mission statement.

Read aloud and analyze the lesson objective to co-create the "I can statement..."
	\begin{itemize}
		\item Duration: 1 class period (approximately 45 minutes)
					\begin{itemize}
						\item Objective:
									\begin{itemize}
										\item Students will understand the concept of a mission statement.
										\item Students will identify the core values and goals of their classroom.
										\item Students will work collaboratively to create a classroom mission statement that aligns
													with the school's mission statement.
										\item Students will develop social-emotional skills, including teamwork, empathy, and
													respect.
									\end{itemize}
						\item Materials:
									\begin{itemize}
										\item Poster paper or chart paper
										\item Markers, colored pencils, or crayons
										\item
													\href{https://www.ps192.org/apps/pages/index.jsp?uREC_ID=1504973&type=d&pREC_ID=1646782}{school mission statement}.
										\item Individual writing journals or notebooks
									\end{itemize}
						\item SEL Skills Learned (Based on SEL Standards):
									\begin{itemize}
										\item Self-Awareness: Students will reflect on their classroom's values and goals.
										\item Self-Management: Students will collaborate and manage their time effectively.
										\item Social Awareness: Students will consider the needs and feelings of their
													classmates.
										\item Relationship Skills: Students will work together to create a shared mission
													statement.
										\item Responsible Decision-Making: Students will make decisions that reflect
													their shared values and goals.
									\end{itemize}
						\item Lesson Outline- Introduction (5 minutes):
									\begin{itemize}
										\item Begin the lesson by asking students if they know what a mission statement is.
													Encourage a brief discussion.
										\item Explain that a mission statement is a statement that describes the purpose,
													values, and goals of an organization, in this case, their classroom.
										\item Show the
													\href{https://www.ps192.org/apps/pages/index.jsp?uREC_ID=1504973&type=d&pREC_ID=1646782}{school mission statement} on the promethean board and read it aloud. Discuss the key
													elements: values, goals, and purpose.
									\end{itemize}
						\item Activity (30 minutes):
									\begin{itemize}
										\item Divide the class into small groups of 3-4 students each.
										\item Distribute individual journals or notebooks to each student for brainstorming.
										\item In their groups, instruct students to brainstorm and write down their ideas for their
													classroom's values, goals, and purpose. Encourage them to think about what makes their
													classroom unique and special.
										\item After brainstorming, have each group choose a spokesperson to share their ideas
													with the whole class. Write these ideas on the whiteboard.
									\end{itemize}
									\newpage
						\item Creating the Classroom Mission Statement (10 minutes):
									\begin{itemize}
										\item As a class, discuss and refine the ideas shared by each group. Encourage students to
													identify common themes and values that resonate with everyone.
										\item Using these ideas, collaboratively craft a classroom mission statement on a large
													sheet of poster paper or chart paper. Emphasize the importance of teamwork and
													respecting each other's ideas.
									\end{itemize}
						\item Conclusion (5 minutes):
									\begin{itemize}
										\item Present the finalized classroom mission statement to the class. Discuss how it
													aligns with the PS192 mission statement.
										\item Discuss the social-emotional skills they used during the activity, such as
													teamwork, empathy, and respect.
										\item Ask students to think about how they can uphold this mission statement in their daily
													classroom interactions.
									\end{itemize}
						\item Assessment:
									\begin{itemize}
										\item Assess students based on their participation in group discussions, their ability
													to collaborate, and their understanding of the concept of a mission statement.
										\item Consider using a rubric that includes criteria like teamwork, creativity, and
													alignment with the school's mission statement.
									\end{itemize}
						\item Extension Activity
									\begin{itemize}
										\item Have students create a visual representation (poster or artwork) of the classroom
													mission statement to display in the classroom.
										\item Note: Adapt the lesson plan as needed to suit the specific grade level and needs
													of your students.
									\end{itemize}
						\item Homework
									\begin{itemize}
										\item Encourage students to write or draw in their individual journals about
													how they will contribute to living up to the classroom mission statement.
									\end{itemize}
					\end{itemize}
	\end{itemize}
\end{document}
