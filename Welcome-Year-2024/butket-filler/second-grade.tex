% Created 2024-09-02 Mon 18:27
% Intended LaTeX compiler: pdflatex
\documentclass[11pt]{article}
\usepackage[utf8]{inputenc}
\usepackage[T1]{fontenc}
\usepackage{graphicx}
\usepackage{longtable}
\usepackage{wrapfig}
\usepackage{rotating}
\usepackage[normalem]{ulem}
\usepackage{amsmath}
\usepackage{amssymb}
\usepackage{capt-of}
\usepackage{hyperref}
\usepackage{minted}
\usepackage{xcolor}
\usepackage{hyperref}
\usepackage{tocloft}
\usepackage[margin=1.8cm]{geometry}
\usepackage{fancyheadings}
\usepackage{minted}
\usepackage[utf8]{inputenc}
\usepackage{amsmath}
\usepackage{amsfonts}
\usepackage{amssymb}
\usepackage{titlesec}
\usemintedstyle{manni}
\usepackage{enumitem}
\usepackage{pdfpages}
\setlength{\parindent}{0cm}
\usepackage{parskip}
\usemintedstyle{friendly}
\usepackage{graphicx}
\usepackage{listings}
\usepackage{float}
\usepackage{colortbl}
\usepackage{booktabs}
\usepackage{wrapfig}
\usepackage{tabularx}
\usepackage{color}
\usepackage{tabularray}
\restylefloat{table}
\usemintedstyle{dracula}
\usepackage[table]{xcolor}
\usepackage{setspace}
\usepackage[none]{hyphenat}
\usepackage{xcolor}
\usepackage{pagecolor}
\definecolor{solarizedBase03}{RGB}{0, 43, 54}
\definecolor{solarizedBase02}{RGB}{7, 54, 66}
\definecolor{solarizedBase01}{RGB}{88, 110, 117}
\definecolor{solarizedBase00}{RGB}{101, 123, 131}
\definecolor{solarizedBase0}{RGB}{131, 148, 150}
\definecolor{solarizedBase1}{RGB}{147, 161, 161}
\definecolor{solarizedBase2}{RGB}{238, 232, 213}
\definecolor{solarizedBase3}{RGB}{253, 246, 227}
\definecolor{solarizedYellow}{RGB}{181, 137, 0}
\definecolor{solarizedOrange}{RGB}{203, 75, 22}
\definecolor{solarizedRed}{RGB}{220, 50, 47}
\definecolor{solarizedMagenta}{RGB}{211, 54, 130}
\definecolor{solarizedViolet}{RGB}{108, 113, 196}
\definecolor{solarizedBlue}{RGB}{38, 139, 210}
\definecolor{solarizedCyan}{RGB}{42, 161, 152}
\definecolor{solarizedGreen}{RGB}{133, 153, 0}
\pagecolor{solarizedBase3}
\color{solarizedBase00}
\hypersetup{
colorlinks=true,
linkcolor=solarizedBlue,
filecolor=solarizedGreen,
urlcolor=solarizedOrange,
citecolor=solarizedMagenta,
}
\titleformat{\section}
{\color{solarizedBlue}\normalfont\Large\bfseries}
{\color{solarizedBlue}\thesection}{1em}{}
\titleformat{\subsection}
{\color{solarizedGreen}\normalfont\large\bfseries}
{\color{solarizedGreen}\thesubsection}{1em}{}
\titleformat{\subsubsection}
{\color{solarizedYellow}\normalfont\normalsize\bfseries}
{\color{solarizedYellow}\thesubsubsection}{1em}{}
\definecolor{draculaBackground}{HTML}{282a36}
\definecolor{draculaForeground}{HTML}{f8f8f2}
\definecolor{draculaSelection}{HTML}{44475a}
\definecolor{draculaComment}{HTML}{6272a4}
\definecolor{draculaCyan}{HTML}{8be9fd}
\definecolor{draculaGreen}{HTML}{50fa7b}
\definecolor{draculaOrange}{HTML}{ffb86c}
\definecolor{draculaPink}{HTML}{ff79c6}
\definecolor{draculaPurple}{HTML}{bd93f9}
\definecolor{draculaRed}{HTML}{ff5555}
\definecolor{draculaYellow}{HTML}{f1fa8c}
\usepackage[table,xcdraw]{xcolor}
\author{Robert Alicea}
\date{\today}
\title{P.S. 192 Family Handbook 2023-24}
\hypersetup{
 pdfauthor={Robert Alicea},
 pdftitle={P.S. 192 Family Handbook 2023-24},
 pdfkeywords={},
 pdfsubject={},
 pdfcreator={Emacs 29.4 (Org mode 9.6.15)}, 
 pdflang={English}}
\begin{document}

\includepdf[pages=1,fitpaper]{pdf1.pdf}

\pagenumbering{\fancyhf{}}
\pagestyle{headings}
\pagenumbering{arabic}

\fancyhead[R]{\thepage}

\fancyfoot[C]{The School of joyful Learning \& Infinite Potential}
\pagestyle{fancy}
\renewcommand{\footrulewidth}{1px}

\definecolor{dkgreen}{rgb}{0,0.6,0}
\definecolor{gray}{rgb}{0.5,0.5,0.5}
\definecolor{mauve}{rgb}{0.58,0,0.82}

\clearpage
\clearpage \tableofcontents \clearpage

\section{Introduction}
\label{sec:org400d7b1}
In the heart of New York City, at a school called PS192, where joyful learning filled every corner, lived two wise lions: Leo and Lea. They were known throughout the school for their courage, fairness, and ability to help others learn. Every student admired them, but one day, something changed…

\section{Story: The Tale of the Wise Lions at PS192}
\label{sec:org346b0e5}

\subsection{Chapter 1: The Mysterious New Student}
\label{sec:orgb2bdfac}
It was the first day of a new school year, and all the students were buzzing with excitement. As everyone settled into their seats, the door creaked open, and in walked a mysterious new student. She was different from anyone they had ever seen before—she had shimmering scales and a tail that swished as she walked.

"Welcome, Sapphire!" Ms. Joy, the teacher, greeted warmly. "Sapphire comes from a very distant place and has traveled a long way to join us. I hope you all will make her feel at home."

Leo and Lea exchanged curious glances. They knew this was the beginning of something important.

\subsection{Chapter 2: The Challenge of Justice}
\label{sec:orgc5e9b50}
During recess, Leo noticed something troubling. A group of students was gathered around Sapphire, whispering and pointing. "She’s so strange," one of them said. "Why does she look like that?"

Leo’s heart sank. He knew this wasn’t fair. He remembered what Ms. Joy had taught them about justice—treating everyone fairly and standing up for what’s right, even when it’s hard.

Leo took a deep breath and approached the group. "Hey, that’s not cool," he said firmly. "Sapphire is our friend, and we should treat her with respect."

The other students looked surprised. They hadn’t meant to be unkind, but sometimes it’s easy to forget how our words can hurt others. "You’re right, Leo," one of them said. "We should give Sapphire a chance."

Leo smiled, and together they invited Sapphire to join their game. The air felt lighter, and Leo knew he had done the right thing. Justice had been served.

\subsection{Chapter 3: The Test of Honor}
\label{sec:orgf6365d8}
As the days went by, Sapphire became more comfortable at PS192. She was kind, smart, and had a special talent for solving puzzles. One day, during a class competition, Ms. Joy announced, "Whoever solves this puzzle first will win a special prize!"

Sapphire was excited. She loved puzzles and knew she had a good chance of winning. But as she worked, she noticed something—one of her classmates, a boy named Max, was struggling. His paper was crumpled, and he looked frustrated.

Sapphire had a choice to make. She could focus on winning the prize, or she could help Max, who really needed it. She remembered what Ms. Joy had said about honor—being truthful, keeping promises, and showing respect. Sapphire wanted to be honorable.

So, she put her puzzle aside and went to Max. "Do you want some help?" she asked kindly.

Max looked up, surprised. "Really? But you’ll lose the competition!"

"It’s okay," Sapphire smiled. "Helping you is more important."

Together, they worked on the puzzle, and although Sapphire didn’t win the prize, something much more valuable happened. Max felt better, and a new friendship was formed. Ms. Joy noticed this and gave both Sapphire and Max a special mention for their honor in the next morning assembly.

\subsection{Chapter 4: The Path of Self-Discipline}
\label{sec:orga966785}
The weeks flew by, and soon it was time for the school’s annual Field Day—a day filled with races, games, and fun. Sapphire was excited, but she was also nervous. She had never been in a race before and wasn’t sure if she could do it.

Leo noticed her fidgeting. "Are you okay, Sapphire?" he asked.

"I’m just worried about the race," she admitted. "What if I can’t do it?"

Leo thought for a moment. "I remember when I was nervous about something. Ms. Joy told me about self-discipline. It’s about being patient, focusing on your goal, and not giving up, even when it’s tough."

Sapphire nodded. "I’ll try to remember that."

When the race began, Sapphire felt her heart pounding. The other runners were fast, and it was easy to feel discouraged. But she kept repeating Leo’s words in her head: "Focus on your goal, don’t give up."

She ran with all her might, focusing on the finish line. Even when she felt tired, she pushed herself to keep going. And when she crossed the finish line, she was amazed—she had finished the race!

She didn’t win first place, but that didn’t matter. She had learned the power of self-discipline, and that was a victory all on its own.

\subsection{Chapter 5: The Secret of the Wise Lions}
\label{sec:org0ffe3dd}
The school year passed quickly, and soon it was the last day of school. Leo, Lea, and Sapphire sat together under a big tree in the playground, reflecting on everything they had learned.

"You know," Sapphire said, "I was really scared when I first came to PS192. But now, I feel like this is my home."

Leo nodded. "We’ve all learned a lot this year. Justice, honor, self-discipline—they’re not just words. They’re what make us stronger and better friends."

Lea smiled. "And being a wise lion isn’t about being perfect. It’s about trying your best to do what’s right."

As they sat together, watching the sun set over the city, they realized the true secret of being wise lions: it’s about making PS192 a place where everyone feels welcome, respected, and loved.

And so, with hearts full of joy, they promised to carry these lessons with them, not just at PS192, but wherever life would take them.

\textbf{The End.}

\section{Classroom Material}
\label{sec:org5a5d1ad}

\subsection{Activity 1: Justice Role-Play}
\label{sec:org895cc33}
\begin{itemize}
\item \textbf{\textbf{Objective:}} Understand the concept of justice.
\item \textbf{\textbf{Materials:}} None.
\item \textbf{\textbf{Instructions:}} In small groups, students act out scenarios where they have to stand up for what’s right, like Leo did for Sapphire.
\item \textbf{\textbf{Discussion:}} Reflect on how it feels to be just and fair.
\end{itemize}

\subsection{Activity 2: Honor Pledge}
\label{sec:org11aa15d}
\begin{itemize}
\item \textbf{\textbf{Objective:}} Encourage honorable behavior.
\item \textbf{\textbf{Materials:}} Paper, crayons.
\item \textbf{\textbf{Instructions:}} Have students create a personal "Honor Pledge" where they promise to be truthful, respectful, and to help others. They can decorate the pledge and share it with the class.
\item \textbf{\textbf{Discussion:}} Discuss what it means to be honorable in different situations.
\end{itemize}

\subsection{Activity 3: Self-Discipline Challenge}
\label{sec:orgd419357}
\begin{itemize}
\item \textbf{\textbf{Objective:}} Practice self-discipline.
\item \textbf{\textbf{Materials:}} Various challenges (e.g., balance a book on your head, stay quiet for a minute).
\item \textbf{\textbf{Instructions:}} Set up stations where students must complete tasks that require self-discipline. Encourage them to stay focused and not give up.
\item \textbf{\textbf{Discussion:}} Talk about how self-discipline helped them complete the challenges and how they can use it in other areas of life.
\end{itemize}

\section{Teacher Plan}
\label{sec:org046dfaf}

\subsection{Week 1: Introduction to Justice}
\label{sec:org118c470}
\begin{itemize}
\item \textbf{\textbf{Day 1:}} Read Chapter 1. Discuss the concept of justice.
\item \textbf{\textbf{Day 2:}} Activity 1: Justice Role-Play.
\item \textbf{\textbf{Day 3:}} Reflect on how students can practice justice in their daily lives.
\end{itemize}

\subsection{Week 2: Understanding Honor}
\label{sec:org05fe9bc}
\begin{itemize}
\item \textbf{\textbf{Day 1:}} Read Chapter 3. Discuss what it means to be honorable.
\item \textbf{\textbf{Day 2:}} Activity 2: Honor Pledge.
\item \textbf{\textbf{Day 3:}} Encourage students to share examples of honor they’ve seen or practiced.
\end{itemize}

\subsection{Week 3: Building Self-Discipline}
\label{sec:orgd141a15}
\begin{itemize}
\item \textbf{\textbf{Day 1:}} Read Chapter 4. Discuss self-discipline and its importance.
\item \textbf{\textbf{Day 2:}} Activity 3: Self-Discipline Challenge.
\item \textbf{\textbf{Day 3:}} Have students set a personal goal that requires self-discipline and share their progress.
\end{itemize}

\subsection{Week 4: Applying the Lessons}
\label{sec:org714a6ad}
\begin{itemize}
\item \textbf{\textbf{Day 1:}} Review the story and key concepts.
\item \textbf{\textbf{Day 2:}} Reflect on how students have grown in justice, honor, and self-discipline.
\item \textbf{\textbf{Day 3:}} Celebrate the students’ achievements and discuss how they can carry these lessons forward.
\end{itemize}
\end{document}
