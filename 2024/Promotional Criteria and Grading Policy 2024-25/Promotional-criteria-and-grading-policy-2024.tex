% Created 2024-07-27 Sat 22:31
% Intended LaTeX compiler: pdflatex
\documentclass[letterpaper, 12pt]{article}
\usepackage[utf8]{inputenc}
\usepackage[T1]{fontenc}
\usepackage{graphicx}
\usepackage{longtable}
\usepackage{wrapfig}
\usepackage{rotating}
\usepackage[normalem]{ulem}
\usepackage{amsmath}
\usepackage{amssymb}
\usepackage{capt-of}
\usepackage{hyperref}
\usepackage{minted}
\usepackage{xcolor}
\usepackage{hyperref}
\usepackage{tocloft}
\usepackage{minted}
\usemintedstyle{manni}
\usepackage{pdfpages}
\usepackage{fancyhdr}
\usepackage{graphicx}
\usepackage[top=1.4in, left=0.5in, right=0.5in, bottom=0.8in]{geometry}
\usepackage[T1]{fontenc}
\usepackage{helvet}
\pagestyle{fancy}
\renewcommand{\headrulewidth}{0pt}
\renewcommand{\footrulewidth}{0pt}
\setlength{\parindent}{0em}
\setlength{\parskip}{1em}
\usepackage{hyperref}
\usepackage {color}
\usepackage {tabularray}
\usepackage{xcolor}
\hypersetup{
colorlinks=true,
linkcolor=blue,
filecolor=magenta,
urlcolor=cyan,
citecolor=green,
pdfborder={0 0 0}
}
\usepackage[most]{tcolorbox}
\author{Hilduara Abreu}
\date{\today}
\title{Promotional Criteria And Grading Policy 2024\\\medskip
\large school leters to parents}
\hypersetup{
 pdfauthor={Hilduara Abreu},
 pdftitle={Promotional Criteria And Grading Policy 2024},
 pdfkeywords={},
 pdfsubject={},
 pdfcreator={Emacs 29.4 (Org mode 9.6.15)}, 
 pdflang={English}}
\begin{document}

\fancyfoot[C]{\setlength{\unitlength}{1in}\begin{picture}(5,0)\put(-1.8,-0.5){\includegraphics[width=8.8in,height=1.3in]{logo-1}}\end{picture}}
\fancyhead[C]{\setlength{\unitlength}{1in}\begin{picture}(5,0)\put(-1.9,-0.5){\includegraphics[width=8.9in,height=1.3in]{logo-2}}\end{picture}}
\fancyhead[R]{\thepage}
\pagenumbering{gobble}

\begin{document}
\vspace*{0.01in}

\newtcolorbox{redbox}[1][]{
  colback=red!5!white,
  colframe=red!75!black,
  fonttitle=\bfseries,
  coltitle=black,
  enhanced,
  attach boxed title to top center={yshift=-2mm},
  title=#1,
  boxed title style={colback=red!50!white}
}

Date: \href{https://www.ps192.org/apps/bbmessages/show\_bbm.jsp?REC\_ID=139439}{September
5th, 2024}

Subject: \textbf{Promotional Criteria and grading Policy 2024-25}

Dear Parents and Guardians,

Chancellor Regulation A-501 implements a system-wide promotion policy with clearly defined standards for
promotion for each grade. The P.S. 192 Promotional Criteria Policy provides the process and procedures
for the implementation of this promotion policy. This policy is effective as of September 5th, 2024.

This policy is being promulgated in the context of the following goals established by the Chancellor’s
Regulation A-501:

All students in Kindergarten through grade 5 will meet or exceed rigorous academic standards in a performance-based core curriculum. In grades 3 through 5, all students will meet or exceed the promotion standards referred to in this regulation, and set forth in DOE issued guidance, in order to be promoted to the next grade and, ultimately, to be prepared for college and careers.

\begin{itemize}
\item The entire school community will be engaged continuously in creating and supporting effective strategies for improved student achievement.
\item A comprehensive student assessment system, aligned with established State and City performance standards, will be used on an ongoing basis to measure student progress and to improve classroom instruction.
\end{itemize}

\begin{redbox}[Classwork Grading System]
\begin{table}[H]
\centering
\begin{tblr}{
  colspec={|X|X|},
  row{1}={font=\bfseries\color{MacaroniandCheese},c},
  hlines,
  vlines,
  hline{1,6} = {-}{0.08em},
}
\textbf{Component}              & \textbf{Weight} \\
In-House Assessments            & 50\%            \\
Daily Classwork                 & 30\%            \\
Classroom Participation         & 10\%            \\
Projects                        & 5\%             \\
Homework                        & 5\%             \\
\end{tblr}
\end{table}
\end{redbox}

\textbf{Promotional Criteria for Grades K-2}
\begin{itemize}
\item 95 percent Attendance
\end{itemize}
\pagebreak
\vspace*{-1cm}
\begin{itemize}
\item Meet Performance Standards in ALL Core Subjects: ELA, Math, S.S., and Science. This means to
obtain a Performance Level 2 (a numeric score of 65 percent) in all core subject areas: Reading,
Writing, Mathematics, Science, and Social Studies. The average of the quizzes and unit exams will
be used to determine the overall grade:
\begin{itemize}
\item Level 1: An aggregate average score of 0-64 points
\item Level 2: An aggregate average score of 65-79 points
\item Level 3: An aggregate average score of 80-89 points
\item Level 4: An aggregate average score of 90-100 points
\end{itemize}
\end{itemize}

\textbf{Reading: Meet Minimum Grade Specific DRA Reading Benchmark}
\begin{itemize}
\item Kindergarten: Benchmark Reading Level 6 (E)
\item First Grade: Benchmark Reading Level 15-16 (L)
\item Second Grade: Benchmark Reading Level 18 (J)
\end{itemize}

\textbf{Writing: Obtain a cumulative Level 2 performance rating in the Writing Portfolio}
\begin{itemize}
\item Kindergarten: 4 Writing Pieces (2 fiction and 2 non-fiction)
\item First grade: 4 Writing Performance Tasks (2 fiction and 2 non-fiction)
\item Second grade: 4 Writing Performance Tasks (2 fiction and 2 non-fiction)
\end{itemize}

\textbf{Math: Obtain a cumulative Level 2 performance rating. The average of the quizzes and unit exams will be used to determine the overall grade.}
\begin{itemize}
\item Level 1: An aggregate average score of 0-64 points
\item Level 2: An aggregate average score of 65-79 points
\item Level 3: An aggregate average score of 80-89 points
\item Level 4: An aggregate average score of 90-100 points
\end{itemize}

\textbf{Project Assignments: Obtain a cumulative Level 2 performance rating in each project.}
\begin{itemize}
\item Kindergarten: 3 Individual Projects (December – S.S.; Feb. – Math; Apr. – Science)
\end{itemize}
\pagebreak
\vspace*{-1cm}
\begin{itemize}
\item First grade: 3 Individual Projects (December – S.S.; Feb. – Math; Apr. – Science)
\item Second grade: 3 Individual Projects (December – S.S.; Feb. – Math; Apr. – Science)
\end{itemize}

\textbf{Teacher’s Recommendation}
\begin{itemize}
\item Holistic analysis and evidence of classwork
\end{itemize}

\textbf{Promotional Criteria for Grades 3-5}
\begin{itemize}
\item 95 percent Attendance
\item Meet Performance Standards in ALL Core Subjects: ELA, Math, S.S., and Science. This means to obtain a Performance Level 2 (a numeric score of 65 percent) in all core subject areas: Reading, Writing,
\end{itemize}

\textbf{Mathematics, Science, and Social Studies. The average of the quizzes and unit exams will be used to determine the overall grade:}
\begin{itemize}
\item Level 1: An aggregate average of 0-64 points
\item Level 2: An aggregate average of 65-79 points
\item Level 3: An aggregate average of 80-89 points
\item Level 4: An aggregate average of 90-100 points
\end{itemize}

\textbf{Reading: Meet Minimum Grade Specific DRA Reading Benchmark}
\begin{itemize}
\item Third Grade: Benchmark Reading Level 34-38 (M-N)
\item Fourth Grade: Benchmark Reading Level 38-40 (O-P)
\item Fifth Grade: Benchmark Reading Level 50 (Q-R)
\end{itemize}

\textbf{Writing: Obtain a cumulative Level 2 performance rating in the Writing Portfolio}
\begin{itemize}
\item Third Grade: 4 Writing Pieces (2 fiction and 2 non-fiction)
\item Fourth Grade: 4 Writing Performance Tasks (1 fiction and 3 non-fiction)
\item Fifth Grade: 4 Writing Performance Tasks (1 fiction and 3 non-fiction)
\end{itemize}
\pagebreak
\vspace*{-0.5cm}
\textbf{Math: Obtain a cumulative Level 2 performance rating. The average of the quizzes and unit exams will be used to determine the overall grade.}
\begin{itemize}
\item Level 1: An aggregate average score of 0-64 points
\item Level 2: An aggregate average score of 65-79 points
\item Level 3: An aggregate average score of 80-89 points
\item Level 4: An aggregate average score of 90-100 points
\end{itemize}

\textbf{Project Assignments: Obtain a cumulative Level 2 performance rating in each project.}
\begin{itemize}
\item Third Grade: 3 Individual Projects (December – S.S.; Feb. – Math; Apr. – Science)
\item Fourth Grade: 3 Individual Projects (December – S.S.; Feb. – Math; Apr. – Science)
\item Fifth Grade: 3 Individual Projects (December – S.S.; Feb. – Math; Apr. – Science)
\end{itemize}

\textbf{Teacher's Recommendation}
\begin{itemize}
\item Holistic analysis and evidence of classwork
\end{itemize}

\textbf{Promotional Criteria for English Language Learners}

English Language Learners will be held to promotional standards based on the
number of years in NYC Public Schools:
\begin{itemize}
\item 1st year ELLs and SIFEs
\begin{itemize}
\item Meet benchmarks in specific subject areas such as Math, S.S., and Science in
their native language.
\end{itemize}
\item 2nd and 3rd year ELLs
\begin{itemize}
\item Score a level 2 in the NYS Math Assessment and make expected gains in the NYSESLAT (51 points within a proficiency level)
\item Score at least a score of 65 percent (Performance Level 2) in a minimum of
three core subject areas.
\end{itemize}
\end{itemize}
\pagebreak
\vspace*{-1cm}
\begin{itemize}
\item 4th year ELLs will be held to the same standards as English Language Proficient Students.
\end{itemize}

\textbf{Promotional Criteria for Special Education Students}
\begin{itemize}
\item Special Education students will be held to the promotion standards stated in the student’s IEP.
\item A student whose IEP does not specify modified promotion criteria will be held to the same standard promotional criteria as General Education Students.
\item Teachers will use all available assessments: standardized tests, performance tasks, ongoing assessments of student work, conference notes, teacher observations, and professional judgment – as a mechanism to improve classroom instruction and to provide parents with detailed information about their child’s academic progress.
\end{itemize}

All promotional criteria are subject to the Principal’s final approval. Parents will also be involved in the decision-making process. Teachers will maintain collections of students' work and formative and summative data that document students' progress toward meeting performance standards and benchmarks. Teachers will be meeting with parents regularly for:
\begin{itemize}
\item Our staff shall employ various communication methods to ensure parents and guardians are consistently informed about their child’s social-emotional and academic development.
\begin{itemize}
\item Zoom or Google virtual conferences
\item Phone conversations
\item Written communication, which includes ClassDojo, email, and text messages, will be utilized to inform parents.
\end{itemize}
\end{itemize}

In Unity,

\includegraphics[width=0.2\textwidth]{hil_signature}

\textbf{Hilduara Abreu}, \textbf{Principal}

\textit{The School of Joyful Learning!}

\href{https://www.ps192.org}{www.ps192.org}
\end{document}
